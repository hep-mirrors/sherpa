\documentclass[a4]{article}
\newcommand{\done}{\mathrm{d}}
\begin{document}
\section{Description of algorithm}
\begin{enumerate}
\item {\tt Evolve(Configuration * config)}\\
  \begin{itemize}
  \item Iterate over all partons in the configuration - if they are switched off ignore them.\\
    {\tt Evolve(Parton * parton)}, see for details below, returns a boolean (true in the
    case of finding a splitting with evolution parameter $t$ above the cut-off $t_0$, $t>t_0$,
    false if no allowed splitting has been found).  It also keeps track of the winning
    splitting {\tt Splitting * p\_winner}, i.e.\ the splitting with the largest $t$ to date.
    If such a splitting exists, ({\tt p\_winner!=NULL}), the respective parton will be split
    according to the parameters stored in it.  Irrespective of whether the splitting is viable
    in {\tt PerformSplitting} (it will always be possible, unless there is a bug in how the
    offsprings of the splitting are inserted into the configuration), the evolution
    parameter of the configuration will be set to the $t$ of the splitting.
  \item Add a final weight to the overall weight, see below.
  \end{itemize}
\item {\tt Evolve(Parton * parton)}\\
  \begin{itemize}
  \item Start with {\tt p\_winner = NULL}, i.e.\ the default is that we do not find a successful
    splitting.
  \item Iterate over all spectators (one in the case of quarks, two in the case of gluons).\\
    For each spectator, produce {\tt Kernels}, a vector of {\tt Kernel}'s, which encapsulate the
    splitting function and the coupling part of the splitting kernel.  They depend on the splitter
    and on the initial/final-state nature of both splitter and spectator.  Create a trial $t$ for
    this combination by calling
    {\tt GenerateTestSplitting(Kernels * kernels, Splitting \& split)}, where a new
    {\tt Splitting * split} is initiated for each splitter-spectator pair.  If successful the
    {\tt Splitting split} will be filled, and if its $t$ is the largest so far it will be kept
    as the winning splitting, {\tt p\_winner}.  If not, {\tt split} will be deleted.
  \item The method returns {\tt true} or {\tt false} dependent on whether we found a winner or not.
  \end{itemize}
\item {\tt GenerateTestSplitting(Kernels * kernels,Splitting \& split)}\\
  \begin{itemize}
  \item {\tt kernels->Integral(split,p\_massselector)} iterates over all allowed splitting kernels,
    and extracts the sum of the integrals of their overestimators, $\sum_K\bar{I}_K$.
    The integrals are given by
    \begin{equation}
      \bar{I}_K \,=\, \frac{C_K\alpha_S(t_0))}{2\pi}\int\limits_0^1\done z\bar{P}_{a\to bc}(z) =
      \frac{C_K\alpha_S(t_0))}{2\pi}\bar{I}_{a\to bc}\,,
    \end{equation}
    where $C_K = C_F = 4/3$ for quarks, $C_K = C_A/2 = 3/2$ for gluons splitting into gluons,
    and $C_K = 1/2$ for gluons splitting into quarks.  $\alpha_S(t_0)$ obviously is the maximum
    of the strong coupling, since we assume $t$ to be the relevant scale for all splittings.
    The overestimated splitting functions are given by
    \begin{eqnarray}
      \bar{P}_{q\to qg}\,=\,\bar{P}_{g\to gg}^{(1)}\,&=&\,\frac{2(1-z)}{(1-z)^2+\kappa_0^2}\cdot(1+\bar{K})\,,\nonumber\\
      \bar{P}_{q\to gq}\,=\,\bar{P}_{g\to gg}^{(2)}\,&=&\,\frac{2z}{z^2+\kappa_0^2}\cdot(1+\bar{K})\,,\nonumber\\
      \bar{P}_{g\to gg}\, &=& \,1\,.
    \end{eqnarray}
    We have two gluon splitting functions into gluons, which will need to be added.  
    $\kappa_0^2$ is the overestimator of the Curci-Furmanski-Petronzio regulator, $t_0/Q^2$,
    and $\bar{K}$ is the overestimator of the soft enhancement function,
    \begin{equation}
      \bar{K} = \left.K\right|_{t=t_0} \,=\,
      \frac{\alpha_S(t_0)}{2\pi}\left[\,C_A\,\left(\frac{67}{18}-\frac{\pi^2}{6}\right)-
        \frac{10}{9}T_Rn_f(t_0)\right]\,.
    \end{equation}
    The $z$-integrals of the splitting functions are given by
    \begin{eqnarray}
      \bar{I}_{q\to qg}\,=\,\bar{I}_{q\to gq}\,=\,
      \bar{I}_{g\to gg}^{(1)}\,=\,\bar{I}_{g\to gg}^{(2)}\,&=&\,
      \log\left(1+\frac{Q^2}{t_0}\right)\cdot(1+\bar{K})\nonumber\\
      \bar{P}_{g\to gg}\, &=& \,1\,.
    \end{eqnarray}
  \item Starting from a given upper scale for the evolution parameter, $T$, new trial evolution
    parameters $t$ emerge as solution of the equation
    \begin{equation}
      \#\exp\left[-\int\limits_{t_0}^T\frac{\done t'}{t'}\sum\limits_K\,\bar{I}_K\right]\,=\,
      \exp\left[-\int\limits_{t_0}^t\frac{\done t'}{t'}\sum\limits_K\,\bar{I}_K\right]\,,
    \end{equation}
    with $\#$ a random number:
    \begin{equation}
      t = T\cdot\exp\left[\frac{\#}{\sum\limits_K\,\bar{I}_K}\right]\,.
    \end{equation}   
  \end{itemize}
\end{enumerate}
\end{document}
