%==================================================
\documentclass[a4paper,fleqn,10pt]{article}
%= definitions ====================================
%= math ===========================================
\usepackage{amsmath}
\usepackage{amssymb}
\usepackage[a4paper,pdfborder={0 0 0}]{hyperref}
\usepackage[all]{hypcap}
\usepackage{array}
\usepackage{calc}
\usepackage{longtable}
\usepackage{multirow}
%= graphics =======================================
\usepackage{axodraw}
\usepackage{pstricks}
\usepackage{graphicx}
\usepackage{xspace}
%\usepackage{graphpap}
%= citations ======================================
\usepackage{cite}
\usepackage{mcite}
%= layout =========================================
\usepackage[format=hang,labelfont=bf]{caption}
\usepackage{sectsty}
\allsectionsfont{\sffamily}
\subsubsectionfont{\mdseries\itshape\large}
\setlength{\parindent}{0mm}
%\setlength{\voffset}{-1cm}
\setlength{\hoffset}{-1.75cm}
\setlength{\textwidth}{16.5truecm}
\setlength{\textheight}{24cm}
\setlength{\topmargin}{0mm}
\setlength{\headheight}{0mm}
\setlength{\headsep}{0mm}
\setlength{\parskip}{1mm}
\setlength{\mathindent}{2ex}
\let\spreprint\empty
\newcommand{\preprint}[1]{\def\spreprint{\protect#1}}
\let\sinstitute\empty
\newcommand{\institute}[1]{\def\sinstitute{\protect#1}}
\makeatletter
\renewcommand{\maketitle}{\begingroup
  \null\thispagestyle{empty}%
    \ifx\spreprint\empty
      \vskip 5ex
    \else
      \flushright\large\spreprint\vskip 2ex
    \fi
    \vskip 5ex
    \flushleft
      {\sffamily\bfseries\huge\@title}\vskip 2ex
      \@author\vskip 2ex
      \ifx\sinstitute\empty
      \else
        {\small\sinstitute}
      \fi
    \vskip 5ex
  \endgroup
}
\makeatother
\renewenvironment{abstract}{\begin{center}
  {\large\sffamily\bfseries Abstract: }
  \begin{minipage}[t]{0.75\textwidth}
}{\end{minipage}\end{center}\vskip 10ex}
\newenvironment{stress}{\vskip 2ex
  \hspace*{2ex}\begin{minipage}{\textwidth-4ex}
  \em}{\end{minipage}\vskip 2ex}
\newcommand{\myfigure}[3]{
  \begin{figure}[#1]
    \begin{center}
      #2\\
      \parbox[t]{\widthof{#2}}{\caption{#3}}
    \end{center}
  \end{figure}
}
\newcommand{\mytable}[3]{
  \begin{table}[#1]
    \begin{center}
      #2\\
      \parbox[t]{\widthof{#2}}{\caption{#3}}
    \end{center}
  \end{table}
}
%= abbreviations ==================================
\newcommand{\MCatNLO}{M\protect\scalebox{0.8}{C}@N\protect\scalebox{0.8}{LO}\xspace}
\newcommand{\HERWIG}{H\protect\scalebox{0.8}{ERWIG}\xspace}
\newcommand{\HERWIGpp}{H\protect\scalebox{0.8}{ERWIG++}\xspace}
\newcommand{\Ariadne}{A\protect\scalebox{0.8}{RIADNE}\xspace}
\newcommand{\POWHEG}{P\protect\scalebox{0.8}{OWHEG}\xspace}
\newcommand{\Sherpa}{S\protect\scalebox{0.8}{HERPA}\xspace}
\newcommand{\Comix}{C\protect\scalebox{0.8}{OMIX}\xspace}
\newcommand{\Apacic}{A\protect\scalebox{0.8}{PACIC++}\xspace}
\newcommand{\Amegic}{A\protect\scalebox{0.8}{MEGIC++}\xspace}
\newcommand{\Rivet}{R\protect\scalebox{0.8}{ivet}\xspace}
\newcommand{\Professor}{P\protect\scalebox{0.8}{rofessor}\xspace}
\newcommand{\FeynRules}{F\protect\scalebox{0.8}{EYN}R\protect\scalebox{0.8}{ULES}\xspace}
\newcommand{\CSS}{C\protect\scalebox{0.8}{SS}\xspace}
\newcommand{\Ahadic}{A\protect\scalebox{0.8}{HADIC++}\xspace}
\newcommand{\Hadrons}{H\protect\scalebox{0.8}{ADRONS++}\xspace}
\newcommand{\Photons}{P\protect\scalebox{0.8}{HOTONS++}\xspace}
\newcommand{\Pythia}{P\protect\scalebox{0.8}{YTHIA}\xspace}
\newcommand{\Jetset}{J\protect\scalebox{0.8}{ETSET}\xspace}
%= definitions ====================================
\long\def\symbolfootnote[#1]#2{\begingroup%
\def\thefootnote{\fnsymbol{footnote}}\footnote[#1]{#2}\endgroup}
\newcommand{\abs}[1]{\left| #1\right|}
\newcommand{\rbr}[1]{\left( #1\right)}
\newcommand{\abr}[1]{\langle #1\rangle}
\newcommand{\cbr}[1]{\left\{ #1\right\}}
\newcommand{\sbr}[1]{\left[ #1\right]}
\newcommand{\beq}{\begin{equation}}
\newcommand{\eeq}{\end{equation}}
\newcommand{\bal}{\begin{align}}
\newcommand{\eal}{\end{align}}
\newcommand{\done}{{\rm d}}
\newcommand{\dtwo}{{\rm d}^2}
\newcommand{\dthree}{{\rm d}^3}
\newcommand{\dfour}{{\rm d}^4}
\newcommand{\order}{\mathcal{O}}
\newcommand{\mc}[1]{\mathcal{#1}}
\newcommand{\DO}{D$0\!\!\!/$ }
\newcommand{\dst}{\displaystyle}
\newcommand{\sst}{\scriptstyle}
\newcommand{\qcut}{Q_{\mathrm{cut}}}
\newcommand{\Nmax}{N_{\mathrm{max}}}
\newcommand{\GeV}{\mathrm{GeV}}
\newcommand{\bV}{{\bf V}}
\newcommand{\bT}{{\bf T}}
\newcommand{\CF}{C_{\mathrm{F}}}
\newcommand{\Nc}{N_{\mathrm{c}}}
\newcommand{\CA}{C_{\mathrm{A}}}
\newcommand{\TR}{T_{\mathrm{R}}}                                               
\newcommand{\q}{\mathrm{q}}
\newcommand{\Q}{\mathrm{Q}}
\newcommand{\qbar}{\mathrm{\overline{q}}}
\newcommand{\Qbar}{\mathrm{\overline{Q}}}
\newcommand{\g}{\mathrm{g}}
\newcommand{\kperp}{k_\perp}
\newcommand{\kperpbf}{{\bf{k}}_\perp}
\newcommand{\kperpbfsq}{{\bf{k}}_{\perp}^{\,\!2}}
\newcommand{\kperpzero}{{\bf{k}}_{\perp,0}}
\newcommand{\kperpzerosq}{{\bf{k}}_{\perp,0}^{\,\!2}}
\newcommand{\kperpmax}{{\bf{k}}_{\perp,{\rm max}}}
\newcommand{\logyc}{\log_{10} (\qcut^{2}/s)}
%==================================================

%= title ==========================================
\hypersetup{
  pdfauthor={},
  pdftitle={Ahadic, a new cluster hadronisation model}
  pdfkeywords={}
}
\preprint{}
\author{Frank Krauss$^{1}$, Jan Winter$^{2}$}
\title{\Ahadic, a new cluster hadronisation model}
\institute{
$^{1}$ Institute for Particle Physics Phenomenology, 
  Durham University, Durham DH1 3LE, UK\\
$^{2}$ Fermi National Accelerator Laboratory, 
  Batavia, IL 60510, USA
}
%= document =======================================
\begin{document}
\maketitle
\begin{abstract}
  The physics of the implementation of a new cluster hadronisation model
  for \Sherpa is presented.  It is based on \cite{Winter:2003tt}.  
\end{abstract}
%= text ===========================================

%=========================================================================
%=========================================================================
%=========================================================================
\section{The \Ahadic model: Overview}
%=========================================================================
%=========================================================================
%=========================================================================
\label{Sec::Model}

\begin{appendix}
%=========================================================================
%=========================================================================
%=========================================================================
\section{Dynamics parameters}
%=========================================================================
%=========================================================================
%=========================================================================
\label{Sec::Dynamics}
In this section the parameters related to the dynamics of gluon decays, 
cluster decays and the transition of clusters into hadrons are discussed.  In 
principle, all decays apart from direct those where a cluster decays directly 
into two hadrons, are governed by a QCD-inspired form.  There, the transverse 
momentum $p_\perp$ and the light-cone momentum $z$ are selected according to a 
probability density, which is motivated by the QCD parton shower picture.  The
crucial difference, however, is that in the parton shower a non-emission is
a viable possibility - this is encoded in the Sudakov form factor.  In a
hadronisation model, gluons {\bf must} split; in fact, in \Ahadic they will 
always split into a $q\bar q$ or a $d\bar d$-pair\footnote{
  Here and in the following diquarks will summarily denoted by $d$, if their 
  flavour content and spin is important, a notation $(qq')_s$ will be adopted.
  Quarks and diquarks will summarily be denoted as flavours.}.  
In addition, the \Ahadic model relies on cluster decays into clusters being 
described in the same spirit: The cluster constituents, i.e.\ flavours, 
{\bf must} emit a (non-perturbative) gluon, which in turn {\bf must}
split.  Therefore, the Sudakov form factor of the parton shower is replaced by
a probability density that looks more like its argument:
\beq
\label{Eq:splittingprob}
\done{\cal P}(t,\,z) \propto 
\done t\done z
\frac{\alpha_S(t)}{\left[t+p_{\perp,0}^2\right]^\eta}
\,\tilde P(z)\,,
\eeq
where $\tilde P(z)$ denotes the, possibly suitably modified, corresponding 
splitting function and $t$ is the scale at which the splitting occurs.  
Typically, this scale is identified with $t=p_\perp^2$, but an option
is available, where in $g\to q\bar q$-splitting this is replaced by the
invariant mass of the flavour pair.

%=========================================================================
%=========================================================================
\subsection{Selection of $p_\perp$}
%=========================================================================
%=========================================================================
\label{Sec::pperpselection}

%=========================================================================
\subsubsection{Naive parametrisations}
%=========================================================================
In all decays, the transverse momentum $p_\perp$ of the decay products is 
selected according to a probability density ${\cal P}$, which in the default
parametrisation is given by
\beq
    {\cal P}(p_\perp^2) \propto 
    \frac{\alpha_s(p_\perp^2+p_{\perp,0}^2)}
         {\left[p_\perp^2+p_{\perp,0}^2\right]^\eta}\,.
\eeq
Here, the parameter $p_{\perp,0}^2$ ensures that the strong coupling is 
suitably regularised and does not diverge.  In addition, in all cases,
the dynamics of the decaying system induces a maximal transverse momentum
available in the decay.  In \Ahadic, there's the option to rescale
the maximally available transverse momentum by a suitable prefactor,
$\kappa_{p_\perp}$.  These two parameters, their tags and default values are 
listed in Table \ref{Tab:PT}.  
\begin{table}[h!]
  \label{Tab:PT}
  \begin{center}
    \begin{tabular}{|c||c|c|}
      \hline
      \multicolumn{3}{|c|}
                  {Parameters steering the dynamics of cluster decays}\\
      \hline
      Physical parameter & Parameter tag & Default\\
      \hline
      \hline
      $\vphantom{\frac||}$
      $\eta$           & \tt{PT\_EXPONENT}      & 1.0\\
      $\vphantom{\frac||}$
      $p_{\perp,0}^2$   & \tt{PT\^{}2\_0}        & 0.25\\
      $\vphantom{\frac||}$
      $\kappa_{p_\perp}$ & \tt{PT\_MAX\_FACTOR}   & 1.0\\
      $\vphantom{\frac||}$
      $\zeta$          & \tt{P\_Q2QG\_EXPONENT} & 1.0\\
      \hline
    \end{tabular}

    \parbox{12cm}{\caption{Parameters for $p_\perp$- an $z$-selection.}}
  \end{center}
\end{table}
In addition, there are a number of parametrisations of the strong coupling, 
accessible through the tag {\tt ALPHAS\_FORM } (default value = 1), with an 
interpretation listed in Table \ref{Tab:asform}.
\begin{table}[h!]
  \label{Tab:asform}
  \begin{center}
    \begin{tabular}{|c||c|}
      \hline
      \multicolumn{2}{|c|}
                  {Switch values for {\tt ALPHAS\_FORM } (default = 1)}\\
      \hline
      Parameter value & Description\\
      \hline
      \hline
      1 &
      \begin{minipage}[ht]{8cm}
        Default $\alpha_S$:\\
        Here the ordinary running strong QCD coupling, with an appropriate 
        number of flavours $n_F$ is used.  In all cases, $n_F\ge 3$.
      \end{minipage}\\
      \hline
      2 &
      \begin{minipage}[ht]{8cm}
        IR-modified $\alpha_S$:\\
        This is the original, default one-loop running QCD strong coupling
        $\alpha_S(Q^2)$, modified with a factor $Q^2/(Q^2+p_{\perp,0}^2)$.  
        This ensures that in the infrared limit, i.e.\ for 
        $Q^2 = p_\perp^2\longrightarrow 0$, the strong coupling and, 
        consequently, the decay probability vanishes.
      \end{minipage}\\
      \hline
      8 &
      \begin{minipage}[ht]{8cm}
        GDH-inspired $\alpha_S$:\\
        Here, the parametrisation of \cite{Deur:2008rf} is being used.  
      \end{minipage}\\
      \hline
    \end{tabular}

    \parbox{12cm}{\caption{Parametrising the form of $\alpha_S$.}}
  \end{center}
\end{table}

%=========================================================================
\subsubsection{Ordering in $p_\perp$}
%=========================================================================
\label{Sec::pperpordering}
Angular (or $p_\perp$-) ordering is a well-known property of QCD, resulting 
from destructive interferences between different gluon emission 
amplitudes\footnote{
  It is not entirely clear, how this applies to gluon splitting into a 
  $q\bar q$-pair.}.
In \Ahadic, there are different options, how strictly this ordering is 
implemented, they are listed in Table \ref{Tab:ptordering}.  These options
can be selected through {\tt PT\_ORDER} (default=3, but could be 1).  
\begin{table}[h!]
  \label{Tab:ptordering}
  \begin{center}
    \begin{tabular}{|c||c|}
      \hline
      \multicolumn{2}{|c|}
                  {Switch values for {\tt PT\_ORDER } (default = 3}\\
      \hline
      Parameter value & Description\\
      \hline
      \hline
      0 & 
      \begin{minipage}[ht]{8cm}
        No $p_\perp$-ordering whatsoever is applied.
      \end{minipage}\\
      \hline
      1 & 
      \begin{minipage}[ht]{8cm}
        $p_\perp$-ordering is applied in gluon emission only, i.e.\ in 
        cluster decays.  The maximal $p_\perp$ scale is then given by the 
        relative transverse momentum of the dipole constituents in the
        lab frame.  
      \end{minipage}\\
      \hline
      2 & 
      \begin{minipage}[ht]{8cm}
        $p_\perp$-ordering is applied in gluon splitting only, i.e.\ in
        the decays of the perturbatively produced gluons from the parton shower
        and in the non-perturbative gluons invoked in the cluster decays.
        In the former case, a dipole picture is invoked where the gluon acts as
        splitter, supplemented with a spectator.  In this case, the maximal 
        $p_\perp$ is given by the relative transverse momentum of the two
        constituents of the dipole, while in the latter case the maximal
        $p_\perp$ allowed is identical to the $p_\perp$ of the previous
        gluon emission.  
      \end{minipage}\\
      \hline
      3 & 
      \begin{minipage}[ht]{8cm}
        $p_\perp$-ordering is applied in gluon emission and splitting only, 
        with the implications as described above.
      \end{minipage}\\
      \hline
    \end{tabular}

    \parbox{12cm}{\caption{Parametrising $p_\perp$-ordering in subsequent 
        decays.}}
  \end{center}
\end{table}

{\bf NOTE:} Up to now I have not implemented any $p_\perp$-ordering 
in the direct decays of a cluster into two hadrons.  If necessary, I
will do this, of course.

%=========================================================================
%=========================================================================
\subsection{Selection of $z$}
%=========================================================================
%=========================================================================
\label{Sec::zselection}

%=========================================================================
\subsubsection{Naive picture}
%=========================================================================
As mentioned above, decays in \Ahadic apart from direct $C\to h_1h_2$ decays, 
are described in a picture similar to that employed in the parton-shower, where
the light-cone momenta $z$ are selected according to a corresponding splitting
function.  Ignoring more subtle mass effects for the moment (their treatment 
is described in Sec.\ \ref{Sec:masseffects} below), the two relevant 
splitting functions in \Ahadic are naively given by
\beq
\begin{split}
  \tilde P_{q\to qg} \;=\;& \left[\frac{1+z^2}{1-z}\right]^\zeta\\
  \tilde P_{g\to q\bar q} \;=\;& 1-2z(1-z)\,. 
\end{split}
\eeq
Here $\zeta$ is a free parameter, set by {\tt P\_Q2QG\_EXPONENT}.  It should 
be stressed that the same splitting function is used for both $g\to q\bar q$
and $g\to d\bar d$ splitting.


%=========================================================================
\subsubsection{Precise picture}
%=========================================================================
In fact, the probability densities are more complicated and read:
\beq
\begin{split}
  \tilde P_{q\to qg} \;=\;& J(z)\cdot 
  \left\{(1-z)\cdot \left[\frac{1}{1.-z+zy} - 
    \frac{\tilde v_{23;1}}{v_{23;1}}\cdot
    \frac{1+z+\frac{m_3^2}{p_2p_3}}{2}\right]
  \right\}^\zeta\\
  \tilde P_{g\to q\bar q} \;=\;& J(z)\cdot
  \frac{1-2[z(1-z)-z_-z_+]}{v_{23;1}}\cdot
  \frac{Q_\perp^2}{Q_\perp^2+\frac{m_2^2+m_3^2}{y}}
    \cdot\frac{1}{\sqrt{\lambda(Q^2,0,m_1^2)}}\,. 
\end{split}
\eeq
Here, a notation has been employed, where parton 1(k) labels the spectator,
parton 2(j) is the emitted object, and parton 3(i) is the splitter/emitter.  
In the case of gluon splitting partons 2 and 3 are therefore the new 
flavour pair, while in gluon emission, the new gluon is labelled with 2.  

The following constants with $Q^2$ the mass squared of the cluster have been 
employed:
\beq
\begin{split}
  Q_\perp^2  & \;=\; Q^2-m_1^2-m_2^2-m_3^2\\
  y         & \;=\; \frac{p_\perp^2-(1-z)^2m_2^2-z^2m_3^2}
                         {z(1-z)Q_\perp^2}\;,\;\;\;
                    y_- \;=\; \frac{2m_2m_3}{Q_\perp^2}\;,\;\;\;
                    y_+ \;=\; 1-\frac{2m_1^2(Q^2-m_3^2)}{Q^2Q_\perp^2}\\ 
  v_{23;2}  & \;=\; \frac{\sqrt{(Q_\perp^2 y)^2-4m_2^2m_3^2}}
                        {yQ_\perp^2+2m_3^2}\;,\;\;\;
                    v_{23;1} \;=\; 
                    \frac{\sqrt{[2m_1^2+Q_\perp^2(1.-y)]^2-4m_1^2}}
                         {Q_\perp^2(1-y)}\;,\;\;\;
                    \tilde v_{23;1} \;=\; 
                    \frac{\sqrt{\lambda(Q^2,m_1^2,m_3^2)}}
                         {Q^2-m_3^2-m_1^2}\\
  z_\pm     & \;=\; \frac{2m_3^2+Q_\perp^2y}
                         {2(yQ_\perp^2+m_3^2+m_2^2)}\cdot
                         \left[1\pm v_{12;1}v_{12;3}\right]
                    \;,\;\;\;
                      p_2p_3 \;=\; \frac{yQ_\perp^2}{2} \\
  J(z)      & \;=\; \frac{(1-y)(Q^2-m_1^2-m_3^2)}
                         {[(Q^2-m_1^2-m_3^2)+\frac{m_3^2+m_{ij}^2}{y}]
                           \sqrt{\lambda(Q^2,m_{ij}^2,m_1^2)}}\\
\end{split}
\eeq

%=========================================================================
\subsubsection{Options in $z$ selection}
%=========================================================================

There are different options for how the splitting functions are used in 
different decays, steered by the tag {\tt Z\_FORM}.  The details can be 
found in Table \ref{Tab:zform}.  It should be noted that, not surprisingly, 
this tag has a considerable influence on the spectra of particles in the 
decay, since it is directly related to the energy distribution of secondaries.  
In \Ahadic, three options are available, which interpolate between all
decay being governed by splitting kernels and all $z$ selected according to
a flat distribution.  The default is given by {\tt Z\_FORM }=2, i.e.\ to 
use splitting kernels for the decays of all leading particles (for a definition
see the Section below, Sec.\ \ref{Sec:leadingparticles}) and a flat
distribution for the decays of secondaries.  
\begin{table}[h!]
  \label{Tab:ptordering}
  \begin{center}
    \begin{tabular}{|c||c|}
      \hline
      \multicolumn{2}{|c|}
                  {Switch values for {\tt Z\_FORM } (default = 2)}\\
      \hline
      Parameter value & Description\\
      \hline
      \hline
      0 & 
      \begin{minipage}[ht]{8cm}
        A flat $z$ distribution for all splittings.
      \end{minipage}\\
      \hline
      1 & 
      \begin{minipage}[ht]{8cm}
        All $z$ in all decays are distributed according to the
        corresponding splitting kernel.
      \end{minipage}\\
      \hline
      2 & 
      \begin{minipage}[ht]{8cm}
        Only emissions off and splitting of leading particles
        are treated with the corresponding splitting function.
      \end{minipage}\\
      \hline
    \end{tabular}

    \parbox{12cm}{\caption{Selection options for $z$.}}
  \end{center}
\end{table}

%=========================================================================
\subsubsection{Note on ``Leading particles''}
%=========================================================================
\label{Sec::leadingparticles}

In \Ahadic particles are called leading if they emerge directly from the 
parton shower, i.e.\ they are, for the purpose of the hadronisation, the 
primary particles.  In the case of gluons, obviously, the splitting into a 
flavour pair deletes the ``leading'' tag.  For quarks, however, the situation 
is different.  For them, the ``leading particle''-property is carried through 
their multiple decays, i.e.\ the emission of a non-perturbative gluon does
not affect this characterisation.

%=========================================================================
%=========================================================================
\subsection{Mass effects}
%=========================================================================
%=========================================================================
\label{Sec::masseffects}
By default mass effects are switched on.  This implies that, due to phase 
space constraints, heavier flavours are more often rejected in gluon splitting 
processes.  The only choice available to users is to select the invariant 
mass of the flavour pair rather than their relative $p_\perp$ as relevant
scale in gluon splitting.   
\begin{table}[h!]
  \label{Tab:masstreatment}
  \begin{center}
    \begin{tabular}{|c||c|}
      \hline
      \multicolumn{2}{|c|}
                  {Switch values for {\tt MASS\_TREATMENT }
                  (default = 3)}\\
      \hline
      Parameter value & Description\\
      \hline
      \hline
      0 & 
      \begin{minipage}[ht]{8cm}
        Use $p_\perp^2$ in Eq.\ (\ref{Eq:splittingprob}).
      \end{minipage}\\
      \hline
      3 & 
      \begin{minipage}[ht]{8cm}
        Use $m_{q\bar q}^2$ in Eq.\ (\ref{Eq:splittingprob}).
      \end{minipage}\\
      \hline
      \hline
    \end{tabular}

    \parbox{12cm}{\caption{Selection options for $z$.}}
  \end{center}
\end{table}

%=========================================================================
%=========================================================================
\subsection{Emergence of primary hadrons}
%=========================================================================
%=========================================================================
\label{Sec::hadronemergence}

At some point, either right at the start or after some decays, clusters
enter a region of comparable low mass.  There they leave the regime of cluster
dynamics and enter the regime of primary hadrons.  In \Ahadic, there is, 
in principle, a competition for clusters to transit either directly to
individual primary hadrons or to decay into pairs of them.  This is
realised as follows:

For each cluster a weight for its transition to a well-defined hadron and for 
its decay into a well-defined pair of hadrons is calculated.  

The two different corresponding regimes are given by the masses of the 
heaviest hadron or pair of hadrons, respectively, available for the (given) 
flavour combination of the cluster.  This mass is compared with  the cluster
mass supplemented with some additive offset.  

In other words: if $m_c+\Delta m_{\rm trans}$, the mass of the cluster plus
the transition offset is smaller than the mass of the heaviest hadron with a
compatible flavour component, then the transition channel is open.  If the
cluster is lighter than the lightest available hadron, typically the cluster 
will have to transit into a light hadron; in this case four-momentum
conservation is guaranteed by reshuffling momenta of neighbouring hadrons or
clusters.  If, on the other hand, $m_c$ is larger than the mass of the lightest
hadron pair available after popping a flavour, or if $m_c+\Delta m_{\rm dec}$ 
is smaller than the mass of the heaviest accessible hadron pair, then the 
channel of the cluster decaying directly into hadrons is open.  If any of the
two channel opens, the cluster will, according to the respective weights,
transit into one or decay into two hadrons.
  
The parameters involved here are listed in Table \ref{Tab:HadronEmergence},
further details can be found in the following.
\begin{table}[h!]
  \label{Tab:HadronEmergence},
  \begin{center}
    \begin{tabular}{|c||c|c|}
      \hline
      \multicolumn{3}{|c|}{Parameters steering the emergence of hadrons}\\
      \hline
      Physical parameter & Parameter tag & Default\\
      \hline
      \hline
      $\Delta m_{\rm trans}$ & {\tt TRANSITION\_OFFSET }    & 0.2 \\
      $\Delta m_{\rm dec}$   & {\tt DECAY\_OFFSET }         & 0.2 \\
      $\alpha$              & {\tt TRANSITION\_EXPONENT }  & 0.5 \\
      $\beta$               & {\tt TRANSITION\_EXPONENT2 } & 0.25 \\
      $\gamma$              & {\tt DECAY\_EXPONENT }       & 0.5 \\
      \hline
    \end{tabular}

    \parbox{12cm}{\caption{Parameters steering the transition of clusters into
        hadrons or their decay into hadrons.  $\Delta m_{\rm trans}$ and
        $\Delta m_{\rm dec}$ are defined in the text, $\alpha$ and $\beta$
        are introduced in Eqs.\ (\ref{Eq:BW}) and (\ref{Eq:transitionweight}), 
        respectively, and $\gamma$ is introduced in Eq.\ 
        (\ref{Eq:decayweight}).}}
  \end{center}
\end{table}

%=========================================================================
\subsubsection{Transition to hadrons}
%=========================================================================
\label{Sec::hadrontransitions}

The detailed weight for the selection of a specific hadron in the transition 
of a cluster to a hadron is given by
\beq
\label{Eq:transitionweight}
{\cal W}_{\rm trans}(C\to H) = 
{\cal W}_{\rm mass}(m_c, m_{\rm had}, \Gamma_{\rm had})\cdot
{\cal W}_{\rm flav}^{\rm transition}\cdot 
\left(\frac{\Gamma_{\rm had}^2}{m_{\rm had}^2}\right)^\beta\,.
\eeq
Here
\beq
{\cal W}_{\rm flav}^{\rm transition} =
\left|\left\langle q\bar q'\left|\right.
      \psi_{\rm had}(q\bar q')\right\rangle\right|^2
\eeq
is the flavour component of the transition probability, depending on the
flavours constituting the cluster and the wave function of the hadron,
$\psi_{\rm had}(q,\,\bar q')$.  The mass-dependent part of the weight is
given by either 1, or a Breit-Wigner distribution:
\beq
\label{Eq:BW}
{\cal W}_{\rm mass}(m_c, m_{\rm had}, \Gamma_{\rm had}) =
\frac{1}{16\pi m_c}\cdot \left\{\begin{array}{ll}
1 & \mbox{\rm (no mass dependence)}\\
\displaystyle{
  \left(\frac{4m_{\rm had}^2}
       {4(m_c-m_{\rm had})^2+\Gamma_{\rm had}^2}\right)^\alpha}
& \mbox{\rm (rel.\ Breit-Wigner)}\\
\displaystyle{
  \left(\frac{m_{\rm had}^4}
       {(m_c^2-m_{\rm had}^2)^2+
         m_{\rm had}^2\Gamma_{\rm had}^2}\right)^\alpha}
& \mbox{\rm (non-rel.\ Breit-Wigner)}
\end{array}\right.
\eeq
\begin{table}[h!]
  \label{Tab:hadrontransitions}
  \begin{center}
    \begin{tabular}{|c||c|}
      \hline
      \multicolumn{2}{|c|}
                  {Switch values for {\tt TRANSITION\_SELECTION }
                  (default = 1)}\\
      \hline
      Parameter value & Description\\
      \hline
      \hline
      1 & 
      \begin{minipage}[ht]{8cm}
        The primary selection is done according to a non-relativistic 
        Breit-Wigner times the wave function overlap.
      \end{minipage}\\
      \hline
      2 & 
      \begin{minipage}[ht]{8cm}
        The primary selection is done according to a relativistic 
        Breit-Wigner times the wave function overlap.
      \end{minipage}\\
      \hline
      3 & 
      \begin{minipage}[ht]{8cm}
        The primary selection is done according to the wave function 
        overlap only.
      \end{minipage}\\
      \hline
    \end{tabular}
    
    \parbox{12cm}{\caption{Choices for the mass part of the weight:
        the function ${\cal W}_{\rm mass}(m_c, m_{\rm had}, \Gamma_{\rm had})$
        in Eq.\ (\ref{Eq:transitionweight}).}}
  \end{center}
\end{table}

%=========================================================================
\subsubsection{Direct decays to hadrons}
%=========================================================================
\label{Sec::hadrontransitions}

The detailed weight for the selection of a specific hadron pair in the 
(of course binary) decay of a cluster is given by
\beq
\label{Eq:decayweight}
{\cal W}_{\rm trans}(C\to H_1H_2) = 
{\cal W}_{\rm mass}(m_c, m_1, m_2)\cdot
{\cal W}{\rm pop}(f)\cdot 
{\cal W}_{\rm flav}^{\rm decay}\cdot 
\left(\frac{4m_1m_2}{m_c^2}\right)^\gamma\,.
\eeq
Here
\beq
{\cal W}_{\rm flav}^{\rm transition} =
\left|\left\langle q\bar q'\left|\right.
      \psi_1(q\bar f)\psi_2(f\bar q')\right\rangle\right|^2
\eeq
is the flavour component of the decay probability, depending on the flavours 
$q$ and $\bar q'$ constituting the cluster and the wave functions of the 
hadron, $\psi_1(q,\,\bar f)$ and $\psi_2(f,\,\bar q')$, where $f$ is the
popped flavour.  The mass-dependent part of the weight is given by the typical
binary decay phase space:
\beq
{\cal W}_{\rm mass}(m_c, m_1, m_2) = \frac{1}{16\pi m_c}\cdot
\frac{\sqrt{[m_c^2-(m_1+m_2)^2][m_c^2-(m_1-m_2)^2]}}{m_c^2}\,.
\eeq

The interplay of the different individual weights in the equation above in 
the construction of the decay weight is steered by the tag
{\tt DECAY\_SELECTION }, (default value = 3) listed in Table 
\ref{Tab:hadrondecays}.
\begin{table}[h!]
  \label{Tab:hadrondecays}
  \begin{center}
    \begin{tabular}{|c||c|}
      \hline
      \multicolumn{2}{|c|}
                  {Switch values for {\tt DECAY\_SELECTION }
                  (default = 3)}\\
      \hline
      Parameter value & Description\\
      \hline
      \hline
      0 & 
      \begin{minipage}[ht]{8cm}
        No decays open.
      \end{minipage}\\
      \hline
      1 & 
      \begin{minipage}[ht]{8cm}
        Phase space only.
      \end{minipage}\\
      \hline
      2 & 
      \begin{minipage}[ht]{8cm}
        Phase space times flavour weight, including the multiplet weights
        and the flavour popping weights.
      \end{minipage}\\
      \hline
      3 & 
      \begin{minipage}[ht]{8cm}
        Phase space times flavour times mass correction weight; here 
        $\gamma$ comes into play.
      \end{minipage}\\
      \hline
    \end{tabular}
    
    \parbox{12cm}{\caption{Choices for constructing the decay weights.}}
  \end{center}
\end{table}

For a discussion, how different weights are included for different multiplets,
cf.\ Sec.\ \ref{Sec::Multiplets}.
%=========================================================================
%=========================================================================
%=========================================================================
\section{Flavour parameters}
%=========================================================================
%=========================================================================
%=========================================================================
\label{Sec::Flavour}
In this section the static parameters related of \Ahadic are discussed.

%=========================================================================
%=========================================================================
\subsection{Constituents and their masses}
%=========================================================================
%=========================================================================
\label{Sec::Constituents}

The constituent quark masses are set by the parameters listed in Table
\ref{Tab:Constituents}
\begin{table}[h!]
  \label{Tab:Constituents}
  \begin{center}
    \begin{tabular}{|c||c|c|}
      \hline
      \multicolumn{3}{|c|}{Parameters steering the constituent masses}\\
      \hline
      Physical mass & Parameter tag & Default\\
      \hline
      \hline
      $\vphantom{\frac||}$
      $m_{g}$ & \tt{M\_GLUE}        & 0.0\\
      $\vphantom{\frac||}$
      $m_{u, d}$ & \tt{M\_UP\_DOWN} & 0.01\\
      $\vphantom{\frac||}$
      $m_{s}$ & \tt{M\_STRANGE}     & 0.2\\
      $\vphantom{\frac||}$
      $m_{c}$ & \tt{M\_CHARM}       & 1.8\\
      $\vphantom{\frac||}$
      $m_{b}$ & \tt{M\_BOTTOM}      & 5.0\\
      \hline
    \end{tabular}

    \parbox{12cm}{\caption{Mass parameters for constituents: quarks and gluon.}}
  \end{center}
\end{table}
The masses of diquarks are calculated according to
\beq
m_{qq'^{(0,1)}} = (m_q + m_q' + \Delta m_{\rm di})\cdot (1+\Delta\mu^{(0,1)})\,,
\eeq
with the parameters of Table \ref{Tab:Diquarks}
\begin{table}[h!]
  \label{Tab:Diquarks}
  \begin{center}
    \begin{tabular}{|c||c|c|}
      \hline
      \multicolumn{3}{|c|}{Parameters steering the diquark masses}\\
      \hline
      Physical parameter & Parameter tag & Default\\
      \hline
      \hline
      $\vphantom{\frac||}$
      $\Delta m_{\rm di}$ & \tt{M\_DIQUARK\_OFFSET} & 0.3\\
      $\vphantom{\frac||}$
      $\Delta\mu^{(0)}$   & \tt{M\_BIND\_0}         & 0.3\\
      $\vphantom{\frac||}$
      $\Delta\mu^{(1)}$   & \tt{M\_BIND\_1}         & 0.4\\
      \hline
    \end{tabular}
    
    \parbox{12cm}{\caption{Parameters for the mass definition of diquarks.}}
  \end{center}
\end{table}


%=========================================================================
%=========================================================================
\subsection{Popping weights}
%=========================================================================
%=========================================================================
\label{Sec::Popping}

In the (enforced) decays of gluons at the end of the parton shower and in 
the decays of clusters, either into clusters or hadrons, quark--\-anti-quark 
or anti-diquark--\-diquark pairs emerge.  In principle this is governed by 
phase space considerations alone, for a discussion of the decay kinematics, 
see Sec.\ \ref{Sec::Kinematics} below.  This implies that, in principle, any 
light flavour degree of freedom (quark or diquark), including spins, comes 
with identical probability.  Hence, typically quarks come with relative
probability 2, spin-0 diquarks with relative probability 1, and spin-1
diquarks are popped with relative probability 3.  

In addition, however, some additional tuning can be provided by specific
flavour popping weights, parametrised by strangeness suppression,
baryon suppression, two weights for diquarks carrying one or two strange
quarks, and for spin-1 diquarks.  These parameters are listed in 
Table \ref{Tab:Popping}.
\begin{table}[h!]
  \label{Tab:Popping}
  \begin{center}
    \begin{tabular}{|c||c|c|}
      \hline
      \multicolumn{3}{|c|}{Parameters steering the popping of flavour}\\
      \hline
      Physical parameter & Parameter tag & Default\\
      \hline
      \hline
      $\vphantom{\frac||}$
      $\kappa_s$  & \tt{STRANGE\_FRACTION}             & 1.0\\
      $\vphantom{\frac||}$
      $\kappa_B$  & \tt{BARYON\_FRACTION}              & 1.0\\
      $\vphantom{\frac||}$
      $\rho_{qs}$ & \tt{P\_{\{QS\}}/P\_{\{QQ\}}}       & 1.0\\
      $\vphantom{\frac||}$
      $\rho_{ss}$ & \tt{P\_{\{SS\}}/P\_{\{QQ\}}}       & 1.0\\
      $\vphantom{\frac||}$
      $\rho_{1}$  & \tt{P\_{\{QQ\_1\}}/P\_{\{QQ\_0\}}} & 1.0\\
      \hline
    \end{tabular}

    \parbox{12cm}{\caption{Additional weights for the popping of flavour in 
      decays of gluons at the end of the parton shower, and in 
      cluster decays.}}
  \end{center}
\end{table}
The total popping weight , i.e.\ the normalisation, then reads
\beq
{\cal N} = 2\left[2+\kappa_s\right] + 
\kappa_B\left[1+2\rho_{qs}+3\rho_1(3+2\rho_{qs}+\rho_{ss})\right]\,.
\eeq
It is straightforward to read off the individual weights/probabilities 
for the popping of a corresponding pair from the equation above.

%=========================================================================
%=========================================================================
\subsection{Hadronic wave functions}
%=========================================================================
%=========================================================================
\label{Sec::Wavefunctions}
I must still fill this section.

%=========================================================================
%=========================================================================
\subsection{Hadronic multiplet weights}
%=========================================================================
%=========================================================================
\label{Sec::Multiplets}
The flavour wave functions of the corresponding hadrons are employed in the 
transition of an individual cluster to single hadron or in the decay of a 
cluster into two hadrons.  Therefore, the wave function part of the
transition or decay weight for a cluster composed of $q\bar q'$\footnote{
  For the sake of a compact notation, diquarks are also denoted by $q$,
  $q'$ etc..  However, it is worth noting that in terms of cluster
  composition etc., the $q$ denote anti-diquarks, while the $\bar q$ denote 
  diquarks.
} 
is given by
\beq
{\cal W}_{\rm flav}^{\rm transition} = 
|\langle q\bar q'|\psi_{\rm had}(q\bar q')\rangle|^2
\;\;\;\mbox{\rm and}\;\;\;
{\cal W}_{\rm flav}^{\rm decay} =
{\cal W}_{\rm pop}(q_{\rm pop})\cdot
|\langle q\bar q'|\psi_{\rm had}(q\bar q_{\rm pop})
                  \psi_{\rm had}(q_{\rm pop}\bar q')\rangle|^2
\eeq
respectively.  Here the $\psi_{q\bar q'}$ denote the wave-function
of the hadron, given in Sec.\ \ref{Sec::Wavefunctions}.  Therefore,
for example, the flavour weight for a $u\bar u$-cluster to transit 
into a neutral pion is given by
\beq
{\cal W}_{\rm flav}(C_{u\bar u}\to\pi^0) = 
\left|\left\langle u\bar u\left|
      \frac{1}{\sqrt{2}}\left(u\bar u-d\bar d\right)\right.
      \right\rangle\right|^2 =
\frac12\,,
\eeq
while the flavour part of the probability of this cluster to decay 
into a charged pion pair is given by
\beq
\label{Eq:C2H}
{\cal W}_{\rm flav}(C_{u\bar u}\to\pi^0) = 
{\cal W}_{\rm pop}(d)\cdot
\left|\left\langle u\bar u\left|
      (u\bar d)(d\bar u)\right.\right\rangle\right|^2 =
{\cal W}_{\rm pop}(d)\,.
\eeq

These weights can be fine-tuned by modifying the a-priori weights for
hadrons of different multiplets to emerge.  In order to switch off
individual multiplets, these weights must be set to zero.  The
hadron multiplets listed in Table \ref{Tab:HadronMultiplets}
are accessible in \Ahadic.
\begin{table}[h!]
  \label{Tab:HadronMultiplets}
  \begin{center}
    \begin{tabular}{|l||l|c|}
      \hline
      \multicolumn{3}{|c|}{Hadron multiplets and their relative weight}\\
      \hline
      Multiplet tag & Light hadrons & Default\\
      \hline
      \hline
      $\vphantom{\frac||}$
      MULTI\_WEIGHT\_L0R0\_PSEUDOSCALARS  & 
      $\pi$, $K$, $\eta$, $\eta'$ & 1\\
      $\vphantom{\frac||}$
      MULTI\_WEIGHT\_L0R0\_VECTORS        &
      $\rho$, $K$, $\omega$, $\phi$ & 1\\
      $\vphantom{\frac||}$
      MULTI\_WEIGHT\_L0R0\_TENSORS2       &
      $a_2(1320)$, $K_2^*(1430)$, $f_2(1270)$, $f'_2(1525)$ & 1\\ 
      $\vphantom{\frac||}$
      MULTI\_WEIGHT\_L0R0\_TENSORS3       &
      $\rho_3(1690)$, $K_3^*(1780)$, $\omega_3(1670)$, $\phi_3(1850)$ & 0 \\
      $\vphantom{\frac||}$
      MULTI\_WEIGHT\_L0R0\_TENSORS4       &
      $a_4(2040)$, $K_4^*(2045)$, $f_4(2050)$, $f_J(2220)$ & 0\\
      $\vphantom{\frac||}$
      MULTI\_WEIGHT\_L1R0\_SCALARS        &
      $a_0(980)$, $K_0^*(1430)$, $f_0(980)$, $f_0(1370)$ & 1\\
      $\vphantom{\frac||}$
      MULTI\_WEIGHT\_L1R0\_AXIALVECTORS   &
      $b_1(1235)$, $K_1(1270)$, $h_1(1170)$, $h_1(1380)$ & 1\\
      $\vphantom{\frac||}$
      MULTI\_WEIGHT\_L1R0\_TENSORS2       &
      $\pi_2(1670)$, $K_2(1770)$, $\eta_2(1645)$, $\eta_2(1870)$ & 0\\
      & \hspace*{1cm}(no heavy hadrons available) & \\
      $\vphantom{\frac||}$
      MULTI\_WEIGHT\_L2R0\_VECTORS        &
      $a_1(1260)$, $K_1(1400)$, $f_1(1285)$, $f_1(1420)$ & 1\\
      $\vphantom{\frac||}$
      MULTI\_WEIGHT\_L3R0\_VECTORS        &
      $\rho(1700)$, $K^*(1680)$, $\omega(1600)$, $f_1^\dagger(1900)$ & 0\\
      & \hspace*{1cm}(no heavy hadrons available) & \\
      $\vphantom{\frac||}$
      MULTI\_WEIGHT\_L0R1\_SCALARS        &
      $\pi(1300)$, $K(1460)$, $\eta(1295)$, $\eta(1475)$ & 0\\
      & \hspace*{1cm}(no heavy hadrons available) & \\
      $\vphantom{\frac||}$
      MULTI\_WEIGHT\_L0R1\_AXIALVECTORS   &
      $\rho(1450)$, $K^*(1410)$, $\omega(1420)$, $\phi(1680)$ & 0 \\
      & \hspace*{1cm}(no heavy hadrons available) & \\
      $\vphantom{\frac||}$
      MULTI\_WEIGHT\_L0R0\_N\_1/2         &
      $p$, $n$, $\Sigma$, $\Xi$, $\Lambda$ & 1\\
      $\vphantom{\frac||}$
      MULTI\_WEIGHT\_L0R0\_N*\_1/2        &
      $N(1440)$, $\Sigma(1660)$, $\Xi(1690)$, $\Lambda(1600)$ & 0\\ 
      $\vphantom{\frac||}$
      MULTI\_WEIGHT\_L1R0\_N*\_1/2        &
      $N(1535)$, $\Sigma(1750)$, $\Xi(1620)$, $\Lambda(1405)$ & 0\\ 
      $\vphantom{\frac||}$
      MULTI\_WEIGHT\_L1R0\_N*\_3/2        
      & \hspace*{1cm}needs to be filled & \\
      $\vphantom{\frac||}$
      MULTI\_WEIGHT\_L0R0\_DELTA\_3/2     &
      $\Delta(1232)$, $\Sigma(1385)$, $\Xi(1530)$, $\Omega^-$ & 1\\
      $\vphantom{\frac||}$
      MULTI\_WEIGHT\_L1R0\_DELTA*\_3/2    &
      $\Delta(1700)$, $\Sigma(1670)$, $\Xi(1820)$, $\Omega(2250)$ & 0\\
      \hline
    \end{tabular}

    \parbox{12cm}{\caption{Hadron multiplets available in \Ahadic.  
      States with a $^\dagger$ are fictitious.}}
  \end{center}
\end{table}
Denoting these weights by ${\cal W}_{\rm multi}$, hence Eq.\ (\ref{Eq:C2H}) 
becomes 
\beq
\begin{split}
{\cal W}_{\rm flav}^{\rm transition} & \;=\;  {\cal W}_{\rm multi}(\rm hadron)\cdot
|\langle q\bar q'|\psi_{\rm had}(q\bar q')\rangle|^2
\;\;\;\mbox{\rm and}\;\;\;\\
{\cal W}_{\rm flav}^{\rm decay} & \;=\;
{\cal W}_{\rm pop}(q_{\rm pop})\cdot
{\cal W}_{\rm multi}(\rm{hadron}_1){\cal W}_{\rm multi}(\rm{hadron}_2)\cdot
|\langle q\bar q'|\psi_{\rm had}(q\bar q_{\rm pop})
                  \psi_{\rm had}(q_{\rm pop}\bar q')\rangle|^2\,.
\end{split}
\eeq
\end{appendix}


%= bibliography ===================================
%\bibliographystyle{bib/amsunsrt_mod}  
%\bibliography{bib/journal}
%= end ============================================
\end{document}
%==================================================
