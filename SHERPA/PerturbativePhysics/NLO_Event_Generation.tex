\documentclass[a4paper,10pt]{scrartcl}

\usepackage[pdfborder={0 0 0}]{hyperref}
\usepackage{fullpage}
\usepackage{amsmath}
\usepackage{eepic}

\newcommand{\done}{{\rm d}}
\newcommand{\nnb}{\nonumber}

%opening
\title{Generating events at NLO}
\date{}

\begin{document}

\maketitle
\tableofcontents
\vspace{2mm}
\hrule
\vspace{5mm}

\section{The total cross section with contrained real emission}

The unconstrained NLO cross section reads
\begin{eqnarray}
 \sigma_{NLO}
& = & \sigma_B + \sigma_V + \sigma_R \nnb\\
& = & \int_N\done\sigma_B
	+ \int_N\done\sigma_V
	+ \int_{N+1}\done\sigma_R \,.
\end{eqnarray}
To render each integral seperately convergent the Catani-Seymour
Dipole Subtraction method is used, thus,
\begin{eqnarray}
 \sigma_{NLO}
& = & \int_N\done\sigma_B
	+ \int_N\left[\done\sigma_V+\int_1\done\sigma_A\right]
	+ \int_{N+1}\left[\done\sigma_R-\done\sigma_A\right] \,.
\end{eqnarray}
If now the real emission is constrained to be within a jet measure $J$,
it is useful to constrain the phase space of the subtraction terms as well.
However, since their integrated version is only available for very simple
phase space constrains $\alpha$, in general not coinciding with the jet
measure $J$, the cross-section can be written as 
\begin{eqnarray}
 \sigma_{NLO,constr.}
& = & \int_N\done\sigma_B
	+ \int_N\left[\done\sigma_V+\int\limits_0^\alpha\done\sigma_A\right] \nnb\\
&&	+ \int_N\int\limits_0^J\left[\done\sigma_R-\done\sigma_A\right]
	+ \int_N\int\limits_\alpha^J\done\sigma_A \,.
\end{eqnarray}
Obviously, the potentially numerically troublesome integration is the last
one over the dipole subtraction terms only. This is due to potentially large
logarithms encountered at both sides of the integration region. Furthermore,
the integrand is strongly peaked towards $\alpha$, if $\alpha < J$.

\subsection{The choice of $\alpha$ for a given jet measure $J$}

The cut-off $\alpha$ used for the dipole splitting functions (presently 
available only for massless splitting functions) is a cut-off in
\begin{eqnarray}
 y = \frac{(p_i\cdot p_j)}{(p_i\cdot p_j)+(p_i\cdot p_k)+(p_j\cdot p_k)} \,.
\end{eqnarray}
where $k$ refers to a spectator (colour partners of combined mother
$\tilde{ij}$ only). The jet measure reads
\begin{eqnarray}
 Q_{ij}^2
& = & 2(p_i\cdot p_j) \min\limits_{k\neq i,j}\frac{1}{C_{i,j}^k+C_{j,i}^k} \,,
\end{eqnarray}
with
\begin{eqnarray}
 C_{i,j}^k
& = & \left\{\begin{array}{cc}
              \frac{(p_i\cdot p_k)}{(p_i+p_k)p_j}-\frac{m_i^2}{2(p_i\cdot p_j)}
		& \mbox{if } j=g \\ 
              1 & \mbox{else}\,.
             \end{array}\right.
\end{eqnarray}

\subsubsection{FF-Dipoles}

\subsubsection*{Massless Case}

With the massless dipole kinematics
\begin{eqnarray}
 y_{ijk}
& = & \frac{(p_i\cdot p_j)}{(p_i\cdot p_j)+(p_i\cdot p_k)+(p_j\cdot p_k)} \\
 z_{i}
& = & \frac{(p_i\cdot p_k)}{(p_i\cdot p_k)+(p_j\cdot p_k)}
\end{eqnarray}
the momenta can be written as (with $(k_\perp\cdot\tilde{p}_{ij,k})$)
\begin{eqnarray}
 p_i
& = & z_i \tilde{p}_{ij} + \frac{\mathbf{k}_\perp^2}{z_i Q^2} \tilde{p}_k + k_\perp \\
 p_j
& = & (1-z_i) \tilde{p}_{ij} + \frac{\mathbf{k}_\perp^2}{(1-z_i) Q^2} \tilde{p}_k - k_\perp \\
 p_k
& = & (1-y_{ijk}) \tilde{p}_k \,.
\end{eqnarray}
The $\tilde{p}_{ij}$ and $\tilde{p}_k$ are the momenta of the leading
order configuration, and
\begin{eqnarray}
 \mathbf{k}_\perp^2
& = & Q^2 y_{ijk} z_i(1-z_i) \\
 k_\perp
& = & (0,\mathbf{k}_\perp) \\
 Q^2
& = & 2(\tilde{p}_{ij}\cdot \tilde{p}_k) \,.
\end{eqnarray}
This gives for the scalar products
\begin{eqnarray}
 (p_i\cdot p_j)
& = & \tfrac{1}{2} y_{ijk} Q^2 \\
 (p_i\cdot p_k)
& = & \tfrac{1}{2} z_i(1-y_{ijk}) Q^2 \\
 (p_j\cdot p_k)
& = & \tfrac{1}{2} (1-z_i)(1-y_{ijk}) Q^2 \,,
\end{eqnarray}
and, hence, for the jet measure
\begin{eqnarray}
 Q_{ij}^2
& = & y_{ijk} Q^2 \min\limits_{k\neq i,j}\frac{1}{C_{i,j}^k+C_{j,i}^k} \,,
\end{eqnarray}
with
\begin{eqnarray}
 C_{i,j}^k
& = & \left\{\begin{array}{cc}
              \frac{z_i(1-y_{ijk})}{1-z_i(1-y_{ijk})} & \mbox{if } j=g \\ 
              1 & \mbox{else}\,.
             \end{array}\right.
\end{eqnarray}
Thus, the optimal value for $\alpha$ for $Q_{ij}^2 < J$ is
\begin{eqnarray}\label{Eq:alpha_opt}
 \alpha_{opt} & = & J/Q^2 \,.
\end{eqnarray}
The last integral above thus comprises the least amount of phase space.
This choice of $\alpha$, however, is dependent on $Q^2$,
\begin{eqnarray}
 \alpha_{opt} \; = \; \alpha(Q^2) \; = \; \alpha(\tilde{p}_{ij},\tilde{p}_k) \,,
\end{eqnarray}
but not on $p_i$, $p_j$ or $p_k$ and, hence, $y_{ijk}$ or $\mathbf{k}_\perp^2$. 
Thus, the integration of the dipole terms proceeds unaffected.

\subsubsection*{Massive Case}

With the massive dipole kinematics
\begin{eqnarray}
 y_{ijk}
& = & \frac{(p_i\cdot p_j)}{(p_i\cdot p_j)+(p_i\cdot p_k)+(p_j\cdot p_k)} \\
 z_{i}
& = & \frac{(p_i\cdot p_k)}{(p_i\cdot p_k)+(p_j\cdot p_k)}
\end{eqnarray}
the scalar products can be written as
\begin{eqnarray}
 (p_i\cdot p_j)
& = & \tfrac{1}{2} y_{ijk} (Q^2 - m_i^2 - m_j^2 - m_k^2) \\
 (p_i\cdot p_k)
& = & \tfrac{1}{2} z_i(1-y_{ijk}) (Q^2 - m_i^2 - m_j^2 - m_k^2) \\
 (p_j\cdot p_k)
& = & \tfrac{1}{2} (1-z_i)(1-y_{ijk}) (Q^2 - m_i^2 - m_j^2 - m_k^2) \,.
\end{eqnarray}
Therefore, the jet measure is
\begin{eqnarray}
 Q_{ij}^2
& = & y_{ijk} (Q^2 - m_i^2 - m_j^2 - m_k^2) \min\limits_{k\neq i,j}\frac{1}{C_{i,j}^k+C_{j,i}^k} \,,
\end{eqnarray}
with
\begin{eqnarray}
 C_{i,j}^k
& = & \left\{\begin{array}{cc}
              \frac{z_i(1-y_{ijk})}{1-z_i(1-y_{ijk})} 
               - \frac{\mu_i^2}{y_{ijk}} & \mbox{if } j=g \\ 
               1 & \mbox{else}\,,
              \end{array}\right.
\end{eqnarray}
with $\mu_i^2=\frac{m_i^2}{Q^2-m_i^2-m_j^2-m_k^2}$.
Since $C_{(i,g);massless} \ge C_{(i,g);massive}$ relation (\ref{Eq:alpha_opt})
holds and an optimal choice for $\alpha$ will be
\begin{eqnarray}
 \alpha_{opt} & = & \frac{J}{Q^2-m_i^2-m_j^2-m_k^2} \,.
\end{eqnarray}
Thus, the massless case can be recovered trivially.

\subsubsection{IF-Dipoles}
\subsubsection*{Massless Case}
\subsubsection*{Massive Case}
\subsubsection{FI-Dipoles}
\subsubsection*{Massless Case}
\subsubsection*{Massive Case}

\subsubsection{II-Dipoles}
\subsubsection*{Massless Case}
The transverse momentum is defined as
\begin{eqnarray}
 \mathbf{k}_\perp^2
& = & 2(\tilde p_{ai}\cdot p_b) v_i\frac{1-X_{iab}-v_i}{x_{iab}} \\
 k_\perp
& = & \left(0,\mathbf{k}_\perp\right)
\end{eqnarray}
with
\begin{eqnarray}
 (k_\perp\cdot \tilde p_{ai})
 (k_\perp\cdot \tilde p_b) = 0 \,.
\end{eqnarray}
The kinematic variables read
\begin{eqnarray}
 x_{iab}
& = & \frac{p_ap_b-p_ip_a-p_ip_b}{p_ap_b} \\
 v_i
& = & \frac{p_ip_a}{p_ap_b} \,.
\end{eqnarray}
Thus, the momenta can be written as
\begin{eqnarray}
 p_a
& = & \frac{1}{x_{iab}}\tilde p_{ai} \\
 p_i
& = & \frac{1-x_{iab}-v_i}{x_{iab}}\tilde p_{ai}+v_i p_b +k_\perp \\
 k_j
& = & \Lambda\left(\tilde p_{ai}+p_b,p_a+p_b-p_i\right)\tilde k_j \,,
\end{eqnarray}
and the spectator $p_b$ remains unchanged while the final state $k_j$
takes the recoil.
The relevant kinematic invariants read
\begin{eqnarray}
 p_i^2 = p_a^2 = p_b^2 = \tilde p_{ai}^2 = 0
\end{eqnarray}
and
\begin{eqnarray}
 (p_i\cdot p_a)
& = & \frac{v_i}{x_iab}(\tilde p_{ai}\cdot p_b) \\
 (p_i\cdot p_b)
& = & \frac{1-x_{iab}-v_i}{x_{iab}}(\tilde p_{ai}\cdot p_b) \\
 (p_a\cdot p_b)
& = & \frac{1}{x_{iab}}(\tilde p_{ai}\cdot p_b) \,.
\end{eqnarray}
Thus, the jet measure of radiation off an II-dipole reads
\begin{eqnarray}
 Q_{ia}^2
& = & 2(p_i\cdot p_a)\frac{2}{C_{i,a}^b+C_{a,i}^b} \nnb\\
& = & 2\frac{v_i}{x_{iab}}(\tilde p_{ai}\cdot p_b)
	\frac{2}{C_{i,a}^b+C_{a,i}^b} \nnb\\
& = & \left\{\begin{array}{ll}
		2\frac{v_i}{x_{iab}}(\tilde p_{ai}\cdot p_b) & \mbox{if }i,a=q \\
		& \\
		2\frac{v_i}{x_{iab}}(\tilde p_{ai}\cdot p_b)\cdot
			2\frac{1-x_{iab}}{2-x_{iab}} & \mbox{if }i=g, a=q \\
		& \\
		2\frac{v_i}{x_{iab}}(\tilde p_{ai}\cdot p_b)\cdot
			2\frac{1+v_i}{2-x_{iab}} & \mbox{if }a=g, i=q \\
		& \\
		2\frac{v_i}{x_{iab}}(\tilde p_{ai}\cdot p_b)\cdot
			\frac{2(1+v_i)(1-x_{iab})}{2(1-x_{iab})+x_{iab}(1-v_i)} & \mbox{if }i,a=g\,.
             \end{array}\right.
\end{eqnarray}
Thus, with the further relation $x_{iab} < 1-v_i$ a suitable $\alpha$
as an integration cut-off in the integral over $\done v_i$ can be found.(?)

Three cases have to be differentiated
\begin{itemize}
 \item $qq$ dipole: In this case the phase space measure reads
	\begin{eqnarray}
	 Q_{ij}^2
	& = & y_{ijk} Q^2 \,,
	\end{eqnarray}
 	thus the unresolved real emission phase space comprises
	$y_{ijk} \in [0,\alpha]$, 
	$z_i \in [0,1]$.
 \item $qg$ dipole: In this case the phase space measure reads
	\begin{eqnarray}
	 Q_{ij}^2
	& = & y_{ijk} Q^2 
		\min\left\{\frac{1-z_i(1-y_{ijk})}
				{z_i(1-y_{ijk})},1\right\} \,,
	\end{eqnarray}
	thus the unresolved real emission phase space comprises
	two regions: $y_{ijk} \in [0,\alpha]$, $z_i \in [0,1]$ 
	and $ y_{ijk} \in [\alpha,\frac{J}{2Q^2}\left(-1+
					\sqrt{1+4\frac{Q^2}{J}}\right)]$, 
	$z_i \in [0,1-\frac{1}{2(1-y_{ijk})}]$.
 \item $gg$ dipole: In this case the phase space measure reads
	\begin{eqnarray}
	 Q_{ij}^2
	& = & y_{ijk} Q^2 
		\min\left\{\frac{1-z_i(1-y_{ijk})}
				{z_i(1-y_{ijk})},\right\} \,,
	\end{eqnarray}
\end{itemize}

\subsubsection*{Massive Case}
 There is no massive case here.





\section{Event generation with constrained real emission}

For POWHEG style event generation the leading order kinematic obtains an NLO
weight according to
\begin{eqnarray}
 w_{NLO}
& = & w_{born} + w_{virtual} + w_{constr. real} \nnb\\
& = & w_B + w_{V+\int_\alpha A} + w_{\int_J[R-A]+\int_\alpha^JA} \,.
\end{eqnarray}
It needs to be seen whether it suffices to sample one extra emission point
per event or the whole constrained extra emission phase space needs to be
integrated per event.




\section{Problems to be resolved}

As Mike pointed out, the projection inherent in $\tilde\Phi_N$ ($N+1\to N$)
may lead to a configuration that will not pass the $N$-parton phase space jet
definition. Thus, although the integrand of
\begin{eqnarray}
 \int_{N+1}\Big[\done\sigma_R-\done\sigma_A\Big]
& = & \int\done\Phi_{N+1} \left|\mathcal{M}_R\right|^2\;F_J(\Phi_{N+1})
	- \int\done\Phi_{N+1} \left|\mathcal{M}_B\right|^2\;
	  \sum\mathcal{D}(\tilde\Phi_N)\;F_J(\tilde\Phi_N)
\end{eqnarray}
exactly vanishes in the singular limit, both terms approach this limit differently.
Hence, the integrand may still contain integrable divergences.

This behaviour is caused by the fact that there exist configurations where the LO
kinematics just pass the jet criterion, but, due to the momentum transfer in
accomodating the extra emission, the NLO configuration, after recombination of the
two unresolved partons, does not. Figure \ref{Fig:malclustering} exemplifies this.
\begin{figure}
 \begin{center}
  \begin{picture}(100,80)(0,0)
   \put(50,35){\vector(0,1){30}}
   \put(50,25){\vector(0,-1){30}}
   \put(47.8,27.5){$\bullet$}
   \dottedline{3}(30,-2)(70,-2)
   \dottedline{3}(30,62)(70,62)
  \end{picture}
  \begin{picture}(10,80)(0,0)
   \put(0,27.8){$\to$}
  \end{picture}
  \begin{picture}(100,80)(0,0)
   \put(50,35){\vector(-1,4){4.5}}
   \put(50,35){\vector(1,4){4.5}}
   \put(50,25){\vector(0,-4){26}}
   \put(47.8,27.5){$\bullet$}
   \dottedline{3}(30,-2)(70,-2)
   \dottedline{3}(30,62)(70,62)
  \end{picture}
  \begin{picture}(10,80)(0,0)
   \put(0,27.8){$\to$}
  \end{picture}
  \begin{picture}(100,80)(0,0)
   \put(50,35){\vector(0,1){34}}
   \put(50,25){\vector(0,-1){26}}
   \put(47.8,27.5){$\bullet$}
   \dottedline{3}(30,-2)(70,-2)
   \dottedline{3}(30,62)(70,62)
  \end{picture}
 \end{center}
 \caption{When creating an extra emission from an $N$ parton state which just
	  barely fulfils the $N$-parton jet criterion there is a possibility
	  that upon creating an extra emission in Catani-Seymour kinematics,
	  affecting also a spectator, and clustering the new pair back into a
	  jet of the $N$-parton phase space the affected spectator no longer
	  fulfils the $N$-parton phase space jet criterion.\label{Fig:malclustering}}
\end{figure}
However, due to first splitting the $N\to N+1$ parton configuration and, then,
evaluating the $N+1$ configuration directly, the (not performed) recombination
is an exact inversion of the splitting, leading to exactly the same $N$ parton
configuration. Thus, the above case can never occur and the cancelation and
subtraction is exact. Nonetheless, this needs to be shown rigorously.

For this point also see S.~Catani, M.~Seymour; Nucl.Phys.B485(1997)291-419,
section 7.1, esp.~the last section: This cancellation mechanism is completely
independent of the axtual form of the jet defining function but it is essential
that $\done\sigma_R$ and $\done\sigma_A$ are proportional to $F_J(\Phi_{N+1})$
and $F_J(\tilde\Phi_N)$, respectively. Nonetheless, both $\done\sigma_R$ and
$\done\sigma_A$ live on the same $N+1$ parton phase space $\done\Phi_{N+1}$.
Thus the numerical integration in $D=4$ dimensions is straightforward. One
generates an $N+1$-parton configuration and gives it a positive weight
($+|\mathcal{M}_R|^2$) or negative weight ($-\sum\mathcal{D}(\tilde\Phi_N)$)
and fills it with the help of the different jet functions $F_J(\Phi_{N+1})$ and
$F_J(\tilde\Phi_N)$ in their respective bin. The large positive and negative
weights then exactly cancel in the singular regions as there both binnings
coincide.

However, when always working in the region of the $N+1$ particle phase space
where $F_J(\Phi_{N+1})$ exhibits its mandatory property,
\begin{eqnarray}
 F_J(p_1,\ldots,p_j=\lambda q,\ldots,p_{N+1})
&\to& F_J(p_1,\ldots,p_{j-1},p_{j+1},\ldots,p_{N+1})\qquad\mbox{ if }\lambda\to 0 \\
 F_J(p_1,\ldots,p_i,\ldots,p_j,\ldots,p_{N+1})
&\to& F_J(p_1,\ldots,p,\ldots,p_{N+1})\qquad\mbox{ if }p_i\to zp,p_j\to(1-z)p \nnb\,,
\end{eqnarray}
then all events are filled in $N$-parton phase space bins and no such distinction
is necessary.



\newpage
\section{CKKW}

LO $N$-parton ME/LO $N+1$-parton ME kernel:
\begin{eqnarray}
 \mathcal{K}_{ab}^{ME(0)}(z,t)
& = & \frac{1}{\sigma_a^{ME(0)}(\Phi_N)}
	\frac{\done^2\sigma_b^{ME(0)}(z,t,\Phi_N)}{\done\log\frac{\mu^2}{t}\;\done z}
\end{eqnarray}
NLO $N$-parton ME/NLO $N$+1-parton ME kernel:
\begin{eqnarray}
 \mathcal{K}_{ab}^{ME(1)}(z,t)
& = & \frac{1}{\sigma_a^{ME(1)}(\Phi_N)}
	\frac{\done^2\sigma_b^{ME(1)}(z,t,\Phi_N)}{\done\log\frac{\mu^2}{t}\;\done z}
\end{eqnarray}
where
\begin{eqnarray}
 \done\sigma_a^{ME(0)}(\Phi_N)
& = & B(\Phi_N) \\
 \done\sigma_a^{ME(1)}(\Phi_N)
& = & B(\Phi_N) + V(\Phi_N) + I(\Phi_N,J)
	+ \int\limits_0^J \done z\done t \Big[R(z,t,\Phi_N)-C(z,t,\Phi_N)\Big]
\end{eqnarray}
It has to be kept in mind, that the normalisation always needs to be at the
same order in $\alpha_s$ as the differential real emission cross-section in
order not to introduce any $K$-factors. It remains to be seen whether the
assumed here implicitely (really?) factorisation of the NLO-kernel really holds.

Regarding the scale choices for $\alpha_s$ for the extra emission, this has to 
be determined in $R(z,t,\Phi_N)$ and, correspondingly, in $C(z,t,\Phi_N)$.
Subsequently, one power of $\alpha_s$ needs to be replaced in $V(\Phi_N)$ and
$I(\Phi_N,J)$ accordingly. To see this consider
\begin{eqnarray}
\lefteqn{
 C_0^{ME}\left[1+\alpha_s
		\left[\tilde V+\tilde I
			+\int\limits_0^J\done z\done t(\tilde R-\tilde C)
		\right]
	\right]
	\Delta(\rho_0,\rho_c)} \nnb\\
& = & C_0^{ME}\left[1+\alpha_s^{ME}
		\left[\tilde V+\tilde I
			+\int\limits_0^J\done z\done t(\tilde R-\tilde C)
		\right]
	\right]
	\left[1-\alpha_s^{PS}\int\limits_{\rho_0}^{\rho_c}\done z\done t\bar R + \ldots\right] \\
& = & C_0^{ME}\left[1+\alpha_s^{PS}
		\left[\frac{\alpha_s^{ME}}{\alpha_s^{PS}}\tilde V
			+\frac{\alpha_s^{ME}}{\alpha_s^{PS}}\tilde I
			+\int\limits_0^J\done z\done t
				\left(\frac{\alpha_s^{ME}}{\alpha_s^{PS}}\tilde R
					-\frac{\alpha_s^{ME}}{\alpha_s^{PS}}\tilde C
					-\bar R \Theta(z,t,\rho_0,\rho_c)
				\right)
		\right]
	\right]
	\tilde\Delta(\rho_0,\rho_c)
\end{eqnarray}
The singularities in $\frac{\alpha_s^{ME}}{\alpha_s^{PS}}\tilde R$ are
still cancelled by $\frac{\alpha_s^{ME}}{\alpha_s^{PS}}\tilde C$, those
in $\bar R$ are excluded from the integration by the $\Theta$-function.
Also, $\frac{\alpha_s^{ME}}{\alpha_s^{PS}}\tilde I - 
\frac{\alpha_s^{ME}}{\alpha_s^{PS}}\int_0^J\tilde C$ still gives exactly zero.



\end{document}
